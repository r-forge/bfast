\documentclass[authoryear,preprint,review,10pt]{elsarticle}
\usepackage{amssymb,amsthm,amsmath,setspace}
\usepackage{url,color}
\usepackage{lineno}
\usepackage[labelfont=bf,format=hang,textfont=it]{caption}
\usepackage{graphicx}
\usepackage{microtype}
\usepackage[letterpaper,text={15cm,23cm}]{geometry}
\usepackage{natbib}
\usepackage{hyperref}

%% for internal use
\newcommand{\fixme}[1]{\emph{\marginpar{FIXME} (#1)}}
\newcommand{\readme}[1]{\emph{\marginpar{README} (#1)}}

\definecolor{Red}{rgb}{0.5,0,0}
\definecolor{Blue}{rgb}{0,0,0.5}
\hypersetup{%
  pdftitle = {Real time change detection},
  pdfsubject = {},
  pdfkeywords = {monitoring, time series, MODIS, NDVI},
  pdfauthor = {Jan Verbesselt, Achim Zeileis},
  %% change colorlinks to false for pretty printing
  colorlinks = {true},
  linkcolor = {Blue},
  citecolor = {Blue},
  urlcolor = {Red},
  hyperindex = {true},
  linktocpage = {true},
}


\journal{Remote Sensing of Environment}

\begin{document}
%% \linenumbers
\begin{frontmatter}

    \title
    {
    Near-real time disturbance detection using satellite image time series
    %Near-real time change detection using satellite image time series:
    %Early warning for forest disturbances
    % Real-time change detection of forest disturbances using satellite image time series
    }
    
    \author[WUR]{Jan Verbesselt\corref{cor}}
    \ead{Jan.Verbesselt@wur.nl}
    \author[UIBK]{Achim Zeileis}
    \ead{Achim.Zeileis@R-project.org}
    \author[WUR]{Martin Herold}
    \ead{Martin.Herold@wur.nl}

    \cortext[cor]{Corresponding author. \emph{Ph}: +31; \emph{Fax}: +31}

    \address[WUR]{Remote Sensing Team, Wageningen University, \\
           Droevendaalsesteeg 3, Wageningen 6708 PB, The Netherlands}
    \address[UIBK]{Department of Statistics, Universit\"at Innsbruck \\
           Universit\"atsstr.~15, A-6020 Innsbruck, Austria}

\singlespace

\begin{abstract}

Monitoring forest disturbances is critical for addressing its impact on carbon
storage, biodiversity, and other socio-ecological processes. Satellite remote
sensing enables cost-effective and accurate monitoring at frequent time steps
over large areas.  Several methods are available to detect disturbances within
historical satellite image time series. However, methods to detect changes
in near real-time within newly captured satellite images are lacking. There is a critical need for
methods that enable rapid analysis of satellite image time series to detect disturbances in near-real time.

We are proposing a generic approach to detect disturbances in near-real time
based on time series modelling of the stable period before the change. As such,
the differentiation between normal and abnormal change in near-real time becomes
possible when new image data is captured. 

The method is a light and fast version based on the ``Break For Additive Seasonal
Trend'' (BFAST) concept. Validation is done (1)~simulating 16-day MODIS NDVI time
series (2000--2010) with different amount of noise, seasonality and containing
disturbances at the end of the time series (2)~by application on real MODIS
satellite image time series to detect forest disturbances in near real-time for
a study area in Australia. 

%Result of the proposed real-time change detection method are compared with currently operational real-time deforestation monitoring system in Brazil.

Preliminary results illustrate that abrupt changes at the end of time series are
successfully detected while being robust for strong seasonality and noise. Cloud
masking however remains important as the clouds can be detected as an abnormal
change. The method will be made publicly available within the BFAST package for
R. 

The proposed method is a automatic and robust change detection approach that can
be applied for different purposes (e.g., fire or oil spil detection, on different
types of data  (e.g., Landsat data and future sensors like the Sentinel
constellation that provide higher spatial resolution at regular time steps). 

\end{abstract}

\begin{keyword}
Early warning \sep Real-time \sep change detection \sep monitoring \sep forest \sep NDVI \sep time series \sep MODIS \sep vegetation dynamics \sep phenology \sep REDD
\end{keyword}

\end{frontmatter} 

\newpage

\section{Concept plan of this paper -- Brainstorm}

\emph{Idea of the paper:}

Focus on the real time change detection method in order to keep the paper short
within a context of forest disturbance detection.  Demonstrate how the method
works using satellite image time series.  Provide examples of how the method
works -- but orient this paper towards a letter for RSE (1)~methods paper
(2)~different applications are possible (forest disturbance, fire detection, oil
spill, etc.). but focus on forest disturbance detection.

\emph{Introduction/Literature review}

\readme{Importance of forest disturbance detection.
The current best system available is the Brazilian DETER system based on
spectral unmixing!!! so we need to compare our systems with their system.... (do
we want to do this because it is a completely different concept==not based on
time series).  (how to they compare unmixed images?) The DETER system
requires landsat images enable the unmixing of the different land cover types
besides a procedure that combines image segmentation, unsupervised
classification, and manual editing to detect deforestation. It is clear that
this method is highly computing intensive and requires expert knowledge.}

Many change detection methods exist but only a few have focussed on real time
change detection using satellite image data
\citep{Shimabukuro:2006vb,White2006}.  
\fixme{ JV: give some examples of near-real time monitoring methods and illustrate their importance. 
the link here between paragraphs is unclear}


None of these methods exploit
the time series properties to differentiate normal (e.g., seasonality, and noise)
from abnormal changes (e.g., disturbances). In previous work, we proposed the
BFAST method which focusses on detect change within historical time series but
which lacks the capacity to detect change in near-real time because a minimum
distance from the end of the time series is required significant trend or
seasonal change detection.

In other science domains like statistics and econometrics, methods exist that
enable online monitoring, i.e., near-real time change detection, using time
series data of e.g., exchange rate regimes \citep{Zeileis:2010tt}. Methods are
developed to detect structural changes within time series, which correspond to
breakpoints at a specific time within a time series and are also called abrupt
trend changes (i.e., abnormal disturbances) within time series
\citep{Verbesselt2009a}. These methods however are not adapted to the typical
time series properties of remotely sensed time series data. Here we will
optimise these methods for remotely sensed time series data (within a context of
forest disturbance detection).

\emph{Concept/What is new?}

We propose a time series analyse approach to detect disturbances in near-real
time (i.e., at the end of time series) within remotely sensed time series data.
Here, we introduce an approach to specifically differentiate between a normal and
abnormal change in near-real time. Abnormal change can be detected by
identifying and modelling a stable period before the change which can be used as
reference for normal variation within the data. By comparing the newly acquired
image -- pixel by pixel -- with a stable historical period the method is able to
detect an abnormal change (i.e., disturbances).

We propose two novel approaches for real-time change detection of remotely
sensed data; (1)~a technique to identify a stable period -- indicating normal
phenological vegetation variation as measured by satellite image time series -
which can be used as reference to differentiate normal from abnormal change
(2)~an approach to compare the stable reference period with the newly acquired images
to detect abnormal change within time series.

This method is developed so that it can be used a global scale since is not computer intensive, does not
require specific thresholds (add reference) and can deal with data gaps in time series (e.g. masked clouds). Furthermore,
the method can be applied onto all sorts of time series data (e.g. in-situ monitoring sensors) or further new Sentinel-2 satellite image time series
which will provide high temporal and high spatial resolution images with global coverage.

, landsat images, segmentation or
unmixing procedures. The method is pixel based and enables the detection of abnormal
disturbance with time series.

\emph{Data}

\readme{I'll start with 16-day data for a study area in Australia but we might need to
test it on 8-day MODIS data (to improve real-time capacity)}

\emph{Scientific closure}

The method is able to detect change in near-real time. 

Noise within satellite image time series can be an issue when identifying the stable period before a
change (especially dealing with daily data). 

Cloud detection algorithms remain important although the method is tested for robustness.
% I am not sure if the method can be used for cloud detection? Especially when using 8-day or daily data
% the influence from clouds will become an issue

The method takes seasonal variation of remote sensed time series of vegetation (land surface
phenology) into account and as such enable detection of abrupt -- abnormal trend
changes (disturbances).

\emph{Journal}

Remote sensing of environment -- Letter? Or other journals? Can we keep it short -- this means that
we have to skip the simulation section and focus on the method and illustrate
how it can be used?  
Environmental Research Letter (ERL)? IEEE? Double check high impact relevant journals.

\readme{the method needs to be compared with the ALERT system used by INPE -> contact
the brazilian/ and talk to Carlos Souza about it. Compare also with rapid
response MODIS data. This will be something for the next paper focussing on the application of the methodology. 
In this paper we will test the method on simulated data and for a small study area for which forest harvest data is available}

\readme{So I will not focus on a specific time series issue (maybe the floods in Darwin,
harvest operation in Green Hillls, something in the Netherlands?, fires in
Greece? -- test this based on NDVI but I can easily be test for other
indices/reflectances)}

\newpage

Here we start working on the final paper -- the section above is the concept of
the paper and the outline.

\section{Introduction}

% forest disturbances and their importance
% the importance of detecting disturbances in near-real time? Disaster monitoring/ fast detection of current forest cover changes/ etc.

Near-real time forest disturbance monitoring is critical for addressing its impact on carbon
storage, biodiversity, and other socio-ecological processes (ref). Deviations from \emph{normal} land surface phenology, defined as the seasonal variation in vegetated land surface from remote sensing \citep{White2009}, can indicate important changes forest health (Christine stone/Morisset2009), carbon status, and even climate change (Cleland et al. 2007) \citep{Hargrove2009}.

\fixme{Add references - Martin any advice on framing this methodological paper?} 

% satellite data
Satellite sensors are well-suited to provide consistent and frequent
measurements over large areas which is appropriate for capturing the effects
of many processes that cause change, including natural (e.g., insect attacks, droughts, fires, floods) and
anthropogenic (e.g., deforestation, urbanisation, farming) disturbances
\citep{Jin2005}. 

% different change types in ecosystem dynamics
Change in ecosystems can be divided into three classes: (1)~\emph{seasonal change}, driven by annual temperature and rainfall interactions impacting plant
phenology; (2)~\emph{gradual trend change} such as trends in mean annual rainfall or gradual change in land
management \citep{Lambin2006}; and (3)~\emph{abrupt trend change}, caused by disturbances such as deforestation and fires. 

% problem statement
Estimating change from remotely sensed data is not straightforward, since time
series contain a combination of seasonal, gradual and abrupt ecosystem changes,
in addition to noise that originates from the sensing environment (e.g., view
angle), remnant geometric errors, atmospheric scatter and cloud effects
\citep{Roy2002,Wolfe1998}. The ability of any system to detect change depends on its capacity to
differentiate normal phenological cycle from abnormal change (e.g. drought stress, degradation, deforestation). An historical perspective is needed to model normal, expected behaviour against which abnormal behaviour can be described \citep{Hargrove2009}.

Several change detection methods are available to detect disturbances within
historical satellite image time series \citep{Coppin2004, deBeurs:2005jq, Verbesselt2009a} but generic methods to detect changes in near real-time within newly captured satellite images are lacking. Two major challenges remain.
%Achim: why do you add extra letters behind each reference? it that done with a certain purpose?


First, change detection techniques need to be independent of region specific thresholds or a change type while being robust against the inherent noise and seasonality captured within time series.
Most change detection methods require user designation of a threshold or change type definition separating real change from spectral changes caused by variability in illumination, seasonality, or atmospheric scattering \citep{Lu2004}.  \citet{White2006} presented a method for real-time monitoring of land surface phenology that avoids problems related to phenological metrics for individual pixels in real-time but requires a region-specific threshold for detecting change. The determination of region specific thresholds adds significant cost to efforts expanding change detection across regions or when regions are changing dynamically.
Trajectory based change detection has been proposed to move towards a threshold independent change detection by characterising change by its temporal signature (Hayes and Cohen, 2007; Kennedy et al., 2007). This approach requires the definition of the change trajectory specific for the type of change to be detected. Furthermore, the method will only function if the observed spectral trajectory matches one of the hypothesised trajectories. This illustrates that there is a critical need for methods that enable rapid analysis of satellite image time series to detect disturbances in near-real time independent of region or data specific thresholds or change types.

% missing data: avoid smoothing and interpolation
Second, time series analysis is needed to differentiate normal from abnormal changes while being able to deal with missing data (e.g. cloud effects or sensor defects). 
Most existing change detection methods smooth the data using one of many existing techniques (e.g. Jonsson Eklundh, 2004, HANTS) when dealing with extremely noisy times series of remotely sensed data. For a given date, these methods typically require looking both backwards and for- wards in time, negating use in real-time or forecast applications \citep{White2006}. Also, time series smoothing and interpolation techniques are unable to deal with abrupt or abnormal changes because the data is modelled and gaps are filled based on assumptions of the normal data variations which inhibits the detection of abnormal changes (i.e. disturbances, deforestation) (Potter et al., 2003). Methods able to exploit and model non-gap filled time series are needed to provide a correct historical perspective to model normal data variation and enable near real-time change detection.

 We propose a generic near-real time change detection concept for time series data. The following major research questions are answered in this paper:

(1)~Can a stable period, representing \emph{normal} historical data variation, be identified within
a time series?

(2)~Is the model representing the \emph{normal stable} historical data variation able to reliably and fast detect disturbances within newly incoming observations (i.e. near real-time)?

We assessed this approach  for a large range of ecosystems by simulating Normalised
Difference Vegetation Index (NDVI) time series with varying amounts of seasonal
variation and noise, and by adding changes with different magnitudes towards the
end of a time series. We applied the approach on MODIS 16-day image composites
(hereafter called 16-day time series) to detect near-real time forest
disturbances in a forested area in south eastern Australia. 

The approach can be used to detect and characterise changes within other
remotely sensed time series (e.g., Landsat, Sentinel sensors) or be integrated within monitoring
frameworks and used as an alarm system to provide information on when and where
significant disturbances occur. The method described in this study are available in the BFAST package for R \citep{R} from CRAN (\url{http://cran.r-project.org/package=bfast})

\section{Real time change detection}\label{sec:Method}

\fixme{Achim could you help here? -- methods section and maybe a few extra
sentences in the intro -- thanks.}

We are using a seasonal trend model similar to the additive season and trend
decompositioning model proposed by \citet{Verbesselt:2010wo}. Here, we are not
decomposing the time series in to a seasonal and trend model but using this
seasonal trend model to assess stability, i.e., normality within a time series,
and detect abnormality at the end of a time series.

It is assumed that seasonal trend model ($Y_t$) is piecewise linear trend and
harmonic model with $m+1$ different segments and $K$ the number of harmonic
terms. Thus, there are $m$ breakpoints $\tau_1^*, \dots, \tau_m^*$ so that:
%
\begin{align} \label{lmod}
  Y_t & = \alpha_i + \beta_i t + \sum_{k=1}^K a_{j,k} \sin\left(\frac{2\pi kt}{f}+\delta_{j,k}\right)  \qquad (\tau_{i-1}^* < t \leq \tau_i^*),
\end{align}
%
where $i = 1, \dots, m$ and we define $\tau_0^* = 0$ and $\tau_{m+1}^* = n$ and
where the unknown parameters are the segment-specific amplitude $a_{j,k}$ and
phase $\delta_{j,k}$ and $f$ is the (known) frequency (e.g., $f=23$ annual
observations for a 16-day time series). We used three harmonic terms (i.e.,
$K=3$) to robustly detect phenological changes within MODIS NDVI time series, as
components four and higher represent variations that that occur on a three-month
cycle or less \citep{Geerken2009,Julien2010}. More information about this
seasonal trend model are provided by \citet{Verbesselt:2010wo}.

As such, the questions mentioned in the introduction can be rephrased into
change detection framework using a seasonal trend model: (1)~Is a given seasonal
and trend model stable within the time period before that changes need to be
detected?  (2)~If it is stable does it remain stable for future incoming
observations? If not this will indicated a abnormal change, i.e., a disturbance.

Here, we embed these questions into a structural change detection framework
where the first question is referred to as testing for structural change (add
references), and the second as monitoring structural change. These questions are
well established for inference about the coefficients in piece-wise
least-squares regression  \citep{Zeileis2003}.

\subsection{Testing for structural change}

\subsection{Monitoring structural change}


\section{Validation}\label{sec:Validation}

The proposed approach can be applied to a variety of time series, and is not
restricted to specific remotely sensed vegetation indices. However, validation
has been conducted using Normalized Difference Vegetation Index (NDVI) time
series, the most widely used vegetation index in medium to coarse scale studies.
The NDVI is a measure of the amount of active photosynthetic biomass, and is
correlated with biophysical parameters such as green leaf biomass and the
fraction of green vegetation cover, whose behaviour follows annual cycles of
vegetation growth \citep{Myneni1995,Tucker1979}.

We validated BFAST by (1)~simulating 16-day NDVI time series, and (2)~applying
the method to 16-day MODIS satellite NDVI time series (2000--2011). Validation
of multi-temporal change-detection methods is often not straightforward, since
independent reference sources for a broad range of potential changes must be
available during the change interval. Field validated single-date maps are
unable to represent the type and number of changes detected \citep{Kennedy2007}.
We simulated 16-day NDVI time series with different noise, seasonality, and
change magnitudes in order to robustly test BFAST in a controlled environment.
However, it is challenging to create simulated time series that approximate
remotely sensed time series, because these contain combined information on
vegetation phenology, interannual climate variability, disturbance events,
sensor conditions (e.g., viewing angle), and signal contamination (e.g., clouds)
\citep{Zhang2009}. Therefore, applying the method to remotely sensed data and
performing comparisons with in-situ data remains necessary. In the next two
sections, simulation of NDVI time series and application on real NDVI time
series are described.

\subsection{Simulation of NDVI time series}

\subsection{Application on real NDVI time series}\label{sec:RealData}

% we use 16-day MODIS data to illustrate the concept and plan to use day and 8-day time series for operational forest disturbance monitoring.


\section{Results}

\subsection{Simulation of NDVI time series}

\subsection{Spatial application on real data}

\section{Discussion and further work}

%More info for other applications of this methodology:
%\url{http://rapidfire.sci.gsfc.nasa.gov/} 
% This might be an interesting link for real time change detection and application if we really want to detect changes fast
% if we really want to detect changes in near-real time we should use the MODIS rapid response option
% this certainly needs to be mentioned in the discussion -- 


% for testing the method we will use 16-day MODIS time series for fit a normal seasonal change pattern (monitoring the stable period) and potentially use daily images to do the abrupt monitoring? or should we stick to the 16-day image at the end of the time series (easier!!! for programming purposes).

\readme{Deviations from �normal� land surface phenologi- cal development can be the first indications of important changes in forest health, including disturbance and recovery (de Beurs and Henebry 2005, Liang and Schwartz 2009, Morisette et al. 2009), carbon status, and even climatic shifts (Cleland et al. 2007).}

!!!!
\readme{
Further, many methods assume that a given mathematical func- tion, such as piecewise logistic functions (Zhang et al., 2003), approximates true phenological development. In global, or even regional application this may be an untenable assumption, es- pecially in cases of disturbance, defoliation, or NDVI curves with sharp peaks or broad plateaus (Potter et al., 2003).}


\readme{adding co-variates e.g. like temperature, precipitation might help differentiating a real change from seasonality.}


\section{Conclusion}

%The approach is adaptable to different remote sensing technologies and provides a foundation for ascribing a sequence of ground conditions (e.g. snowmelt, vegetative growth, pollen production, insect phenology) to remotely sensed land surface phenology observations \citep{White2006}.

\section{Acknowledgements}

This work was undertaken using data available via the program of the Cooperative Research Center for Forestry: Monitoring and Measuring (\url{http://www.crcforestry.com.au}). 

%Thanks to xx whose comments greatly improved this paper. We greatly appreciate the constructive feedback we have received from the three reviewers.


\bibliographystyle{model5-names}
\bibliography{refs}

\newpage

\section*{Figures}

%% (For interpretation of the references to color in this figure legend, the reader is referred to the web
%% version of this article.)

%\begin{figure}[htp]
%\centering
%    \includegraphics[height=0.6\textwidth]{figs/Similarity_Simulated_RealData}
%  \caption{Real and simulated 16-day NDVI time series of a grassland (top) and pine plantation (bottom).}
%  \label{fig:SimGF}
%\end{figure}

\end{document}

